\documentclass[a4paper,10pt]{article}

\usepackage[utf8]{inputenc}
\usepackage[T1]{fontenc}
\usepackage{graphicx}
\usepackage{fullpage}
\usepackage{amsmath}

\author{Germain Pham}
\title{Calcul du SNR en fonction de la distance}
\date{Avril 2021}

\begin{document}
   \maketitle

\begin{align}
    SNR_{lin} 
    &= \frac{P_{sig}}{P_{bruit}}\\
    &= \frac{
        P_{t}G_{t}G_{r}\left(\frac {\lambda }{4\pi d}\right)^{\alpha}
    }{P_{bruit}}\\
    &= \frac{P_{t}G_{t}G_{r}}{P_{bruit}} \left(\frac {\lambda }{4\pi }\right)^\alpha \left(\frac {1 }{ d^\alpha}\right)
\end{align}

d'où :
\begin{align}
    SNR_{dB} 
    &= 10\log_{10}(SNR_{lin}) \\
    &= 10\log_{10}\left( \frac{P_{t}G_{t}G_{r}}{P_{bruit}} \right) 
    + 10 \alpha \log_{10}\left( \frac {\lambda }{4\pi } \right) 
    - 10 \alpha \log_{10}(d)
\end{align}

Ainsi, on peut écrire : 
\begin{equation}
    SNR_{dB} = A \times X + B
\end{equation}
avec
\begin{equation}
    \begin{cases}
        A = 10 \alpha\\
        X = \log_{10}\left( \frac {\lambda }{4\pi } \right)  - \log_{10}(d)\\
        B = 10\log_{10}\left( \frac{P_{t}G_{t}G_{r}}{P_{bruit}} \right) 
    \end{cases}
\end{equation}

\end{document}
